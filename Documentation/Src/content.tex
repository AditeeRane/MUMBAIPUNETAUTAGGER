\section{Installing Standard Programs:}

\subsection{Installing {\tt HepMC}}
{\tt
wget -c '\url{http://lcgapp.cern.ch/project/simu/HepMC/download/HepMC-2.06.09.tar.gz}'\\
tar -xf 'HepMC-2.06.09.tar.gz'\\
cd 'HepMC-2.06.09/'\\
'./configure' '--with-momentum=GEV' '--with-length=MM'\\
make -j4\\
sudo make install\\
}

\subsection{Installing {\tt Pythia8}:}
{\tt
wget -c '\url{http://home.thep.lu.se/~torbjorn/pythia8/pythia8230.tgz}'\\
tar -xf 'pythia8230.tgz'\\
cd 'pythia8230/'\\
./configure '--with-hepmc2'\\
make -j4\\
sudo make install
}

\subsection{Installing {\tt Root-6}}
Download the binary distribution from \url{https://root.cern.ch/content/release-61206} for the appropriate platform and extract.
Once you have extracted, you will get a directory "root" wherever you extracted, cd to this directory and run
{\tt source root/bin/thisroot.sh} (assuming you are using bash shell, otherwise modify accordingly). If you are installing {\tt root} to a non standard directory, it is preferable to put "source root/bin/thisroot.sh" into your {\tt "~/.bashrc"} (or some other initialization file for your shell).

If this method does not produce a good executable then you should compile root from source, please follow the instructions from the official (\url{https://root.cern.ch/building-root}) site (please do not use non-standard scripts / commands from random forums).

\subsection{Installing {\tt Delphes}}
{\tt
	wget -c '\url{http://cp3.irmp.ucl.ac.be/downloads/Delphes-3.4.1.tar.gz}'
	tar -xf './Delphes-3.4.1.tar.gz'\\
	cd './Delphes-3.4.1/'\\
	'./configure'\\
	make -j4	
}\\
Delphes does not have any "make install" but all the executables you need are produced in the source directory and can be copied anywhere required.
An important step required is to manually copy "libDelphes.so" to '/usr/local/lib' or any other library search path and also copy all the root dictionaries
({\tt 'ClassesDict\_rdict.pcm', 'ExRootAnalysisDict\_rdict.pcm', 'FastJetDict\_rdict.pcm',\\ 'ModulesDict\_rdict.pcm'}) to the root dictionary search path and copy the header files ({\tt 'classes', 'external'} folders) to the default headers search paths ({\tt '/usr/local/include'}).